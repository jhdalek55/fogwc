%!TEX root = paper.tex
\section{System Architecture}

We discuss in this section the governing principles behind Seattle's
segmentation of functionality into self-sufficient yet interoperable
components.
An important insight for Seattle as a platform is how the classical
dichotomy of platform operators and users gives way to multi-faceted
trust relationships between a large number of mutually-unrelated
stakeholders.
In Seattle, the actual edge-based software installs, core infrastructure
services, clearinghouse operations, platform software builds, and remote
application deployments might be carried out and managed by different
groups of people, potentially untrusted (or even unknown) between groups
and among members of the same group.
% Then why/how do people trust each other in such a system?


\subsection{Goals And Design Principles}

From the above we can see that the system architecture must be able
to keep the needs of the various parties apart. To facilitate this,
components are designed along these trust boundaries.

Not all components are strictly required for Seattle's functioning.



\albert{...for breaking things into interoperable components. These are a subset of foggish goals (which might include performance isolation etc.)}


\albert{Roughly: 1, separate along trust boundaries. 2, make parts self-sustainable and useful by themselves. 3, stable inter-component interfaces. 4, flexible UIs.}


\subsection{Components}
\albert{These guys implement additional foggish goals.}

