%!TEX root = paper.tex
\section{System Architecture}

An important insight for Seattle as a platform is how the classical
dichotomy of service platform operators and users gives way to multi-faceted
trust relationships between a large number of mutually-unrelated
stakeholders.
In Seattle, the actual edge-based software installs, core infrastructure
services, clearinghouse operations, platform software builds, and remote
application deployments might be carried out and managed by different
groups of people, potentially untrusted (or even unknown) between groups
and among members of the same group.
% Then why/how do people trust each other in such a system?

This gives rise to a set of components~\cite{Cappos2009},
any of which is useful
by itself, that can be freely combined to implement various
deployment models, from pure peer-to-peer operation up to a
provisioned deployment by a dedicated operator. Furthermore,
new components may be introduced freely to replace or augment
existing ones, as long as the component interfaces are adhered to.
We briefly introduce the default components below.
\albert{And yes, they are widely deployed! See section such-and-such!}

\begin{figure}
  \centering
  \includegraphics[width=\columnwidth]{figures/arch.pdf}
  \caption{Architecture components of Seattle}
  \label{fig:arch}
\end{figure}



Figure~\ref{fig:arch} overviews Seattle's components and interfaces.
%The components fall into three categories: device software,
%infrastructure, and control.

We first introduce the components running on all devices that
run Seattle.
The core element of computation is a restricted,
performance- and security-isolated Python-based
\textit{Sandbox}~\cite{RepySandbox,li2015fence}.
The sandbox \gls{API} provides
networking, file, threading, etc. functions to
experiments\footnote{
Following Seattle's research and educational background, we
call code that is executed in sandboxes an ``experiment'',
regardless of its actual nature.}.

The \textit{Resource Manager} isolates % chroots, effectively!
sandboxes that run in
parallel, and interfaces with them on behalf of authorized remote
parties to start or stop them, upload and download data, reset
sandboxes, or transfer ownership. (The sandboxes on one
device might all have different authorized users if so desired,
so that one Seattle node can serve multiple experimenters.)

The \textit{Device Manager} % (installer with \texttt{--percentage}, start/stop scripts, uninstaller, softwareupdater): device owner retains control!
is the device owner's interface to enable or disable Seattle on
the device, and choose the amount of resources that it may consume.

A \textit{Software Updater} keeps all components of the
device software up to date. This concludes the overview of
device-side Seattle components.
\\

To control the fundamental functions of sandboxes (such as
starting to execute code, or downloading collected data),
an \textit{Experiment Manager} % (seash)
is used. The experiment manager contacts sandboxes through the
resource manager interface programmatically, and provides a
human-usable interface to the experimenter. The experimenter
authenticates against sandboxes by using cryptographic keys.

Every experimenter runs an experiment manager, and possesses
their own set of cryptographic keys. This enables interesting
deployment options (and thus applications):
Two experimenters could choose to mutually authenticate the
other on the sandboxes they control, or swap resources, and
so on; making more sandboxes available to an experiment than
a single experimenter could ever reach.
\\

Installer builder: where you get installer from; lets you plant pubkeys.

This sums up all of the Seattle components required for a self-hosted,
infrastructure-less deployment already.


Clearinghouse: well....

Lookup service: a centralized meta-meeting point.



