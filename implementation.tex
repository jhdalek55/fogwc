%!TEX root = paper.tex
\section{Implementation and Practical Deployment}

In this section we will look at the actual implementations of
selected components, with a focus on those aspects that help
Seattle solve the node and operational heterogeneity
challenges encountered in \gls{fc}.

\subsection{Sandbox Implementation}

% Inspired by the repy_v2/README that I wrote a few weeks back.
Seattle's Python-based default sandbox~\cite{RepySandbox}
offers a cross-platform portable, resource-isolated,
safe execution environment for untrusted experimenter code.

The choice of a high-level programming language environment trades
off some reduction in performance against large gains in portability
across \glspl{OS} and devices. Seattle's platform abstractions
provide a unified \gls{API} to sandboxes on desktops and laptops,
but als smartphones and even WiFi routers. Seattle sandboxes
run on Windows, Mac OS X and later, Linux, the \glspl{BSD}, as
well as Android and OpenWrt / LEDE.

Seattle's resource isolation scheme~\cite{li2015fence} ensures
that each sandbox is confined to strict usage quotas for all
resources of the hosting system, including \gls{CPU} time and
memory, used disk space, and even \gls{IP} addresses and port numbers
on network interfaces.

In terms of code safety, Seattle keeps buggy or deliberately destructive
experimenter code from harming the host machine. This is guaranteed
by first checking the static code safety and forbidding potentially
problematic statements (like importing arbitrary modules). At runtime,
all of the \gls{API} functions strictly check their call parameters
(so that, e.g., file names can be sanitized and restricted to the
sandbox directory).

There exist other sandbox implementations that augment Seattle's.
Sensibility Testbed adds sensor functions to Seattle's Python-based
sandbox; an internal research prototype of ours interfaces the
Seattle Resource Manager with Docker, so that its sandboxes are
Linux Containers.




\subsection{Component Interface Implementation}

At the core of addressing operational heterogeneity (see
Section~\ref{such-and-such}) lie clearly-defined interfaces
that use simple, mostly stateless protocols.

Postel's Robustness Principle~\cite[\S2.10]{rfc793}.


\subsection{To Do}
\albert{Rather ``Random Leftover Notes'' I guess.}

Mention device manager and practical problems like NAT.

Mention the clearinghouse because we have other clearinghouses
to talk about later.

\subsection{Community Contributions}

