%!TEX root = paper.tex
\section{Introduction And Related Work}

In recent years the term ``fog computing'' has been established to describe
cloud computing at the edge of the network. Fog networks are typically
characterized by the large number of geographically distributed nodes, ranging
from embedded systems, e.g. network connected
sensors, to smartphones as well as end-user laptops.
~\cite{Bonomi:2012:FCR:2342509.2342513,Yi:2015:SFC:2757384.2757397}
While leveraging the
ubiquitous availability and specific capabilities of such devices brings
many opportunities, it also bears new challenges unknown to traditional cloud
services. Some of these challenges include the heterogeneity of fog nodes,
their accessibility, e.g. behind private networks, and extended security and
privacy requirements.~\cite{botta_integration_2016}
Opportunities as well as the challenges have been widely addressed in the
respective literature, however implementations, let alone actively deployed and
publicly available platforms that can be used in the fog context, are rare. \\ \\

For instance \cite{bellavista_feasibility_2017} investigates the applicability of
docker containers for fog computing. The work uses and extends the Open Source Kura
framework to create IoT gateways that control the information
flow between fog nodes and the cloud, while reviewing and benchmarking different
container related technologies used on the fog nodes.
The focus of the work tends towards IoT computing, where tailored services run
on fog nodes to gather data that is forwarded to to cloud for further processing.
A similiar evaluation of docker containers for edge computing can be found in
\cite{ismail_evaluation_2015}.
The recently released  also addresses fog-node to cloud
communication.

Others works have contributed fog platforms for very specific use cases, e.g.
a service architecture for an emergency alert system, where emergency alerts
issued by fog nodes (i.e. end-user smart-phones) propagate to an ``appropriate''
emergency department \cite{7134091}; or a so-called Adaptive Operations Platform
to improve the effectiveness of applying equipment failure models in the context
of industrial IoT. \cite{gazis_components_2015}.
On the other hand there are implementations that address specific individual
fog computing issues, like Dsouza et al. who propose a framework for
policy management to authenticate the various actors in smart transportation
system \cite{dsouza_policy-driven_2014}.

Additionally, there are proprietary business solutions like Cisco's IOx or
Google IoT cloud.
\lukas{do we care or those?}

\lukas{Albert suggests to also reference \cite{belli_design_2015}}

In contrast, Seattle does not only specify a flexible platform architecture,
is also has over 8 years of experience of operating a general-purpose
infrastructure used among others for network research and education.

The contributions of this paper are as follows:
We present Seattle, a practical platform for \gls{fc} research
with a real-world deployment on heterogeneous nodes,
including desktop and laptop machines, Android devices,
and routers and embedded devices running OpenWrt.
%Seattle nodes host Python-based \glspl{VM} that can run
%general-purpose code on a variety of platforms,
%and which self-isolate so as to not
%affect the safety and performance of the host node.
Seattle's system architecture caters to a variety of use cases,
ranging from peer-to-peer deployments to full-fledged
provisioning by a dedicated operator, and cooperative setups,
where different stakeholders federate multiple parallel running
instances of services.
Our own deployment of Seattle has been used in multiple contexts
including research~\cite{li2015fence,rafetseder2013sensorium,zhuang2014sensibility,Eisl1010:Service,Tuts1010:Sustained,collares2011smart,zhuang2015privacy,cappos2014blursense,7133607} and education~\cite{Wallace_CCSC_2011,Cappos_CCSCCP_2010,Cappos_CCSCNW_2009,Cappos_SIGCSE_2014,Hooshangi_SIGCSE_2015};
other groups have successfully reused and augmented Seattle components
for their purposes~\cite{chard2010social,chard12ssc,caton2014social,muller2014tomato,tomato,eittenberger2012doubtless,zhuang2012distributed,zhuang2014taking,tredger2013building}.
