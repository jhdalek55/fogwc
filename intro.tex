%!TEX root = paper.tex
\section{Introduction And Related Work}

The contributions of this paper are as follows:
We present Seattle, a practical platform for \gls{fc} research
with a real-world deployment on heterogeneous nodes,
including desktop and laptop machines, Android devices,
and routers and embedded devices running OpenWrt.
%Seattle nodes host Python-based \glspl{VM} that can run
%general-purpose code on a variety of platforms,
%and which self-isolate so as to not
%affect the safety and performance of the host node.
Seattle's system architecture caters to a variety of use cases,
ranging from peer-to-peer deployments to full-fledged
provisioning by a dedicated operator, and cooperative setups
where different stakeholders federate multiple parallel running
instances of services.
Our own deployment of Seattle has been used in multiple contexts
including research~\cite{li2015fence,rafetseder2013sensorium,zhuang2014sensibility,Eisl1010:Service,Tuts1010:Sustained,collares2011smart,zhuang2015privacy,cappos2014blursense,7133607} and education~\cite{Wallace_CCSC_2011,Cappos_CCSCCP_2010,Cappos_CCSCNW_2009,Cappos_SIGCSE_2014,Hooshangi_SIGCSE_2015};
other groups have successfully reused and augmented Seattle components
for their purposes~\cite{chard2010social,chard12ssc,caton2014social,muller2014tomato,tomato,eittenberger2012doubtless,zhuang2012distributed,zhuang2014taking,tredger2013building}.
